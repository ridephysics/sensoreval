\documentclass[titlepage]{article}
\usepackage{amsmath}
\usepackage{mathtools}
\usepackage{tikz}
\usetikzlibrary{angles,quotes}
\numberwithin{equation}{section}

\begin{document}

\title{Equations of Motion: Booster}
\author{Michael Zimmermann}
\date{}
\maketitle

\section{Diagram}
\begin{tikzpicture}
% coordinate system
\draw[thick,->] (-5,0) -- (5,0) node[anchor=north west] {x};
\draw[thick,->] (0,-5) -- (0,5) node[anchor=south east] {y};
\draw[fill=black] (0,0) circle (0.1cm) node[label={[above,label distance=0.5cm]east:{$(0, 0)$}}]{};

% rods
\draw (0,0) -- (4,-2) node[pos=0.5,fill=white]{$r_B$};
\draw (4,-2) -- (7,-2) node[above,pos=0.5]{$r_i$};
\draw[fill=black] (4,-2) circle (0.1cm);

% angles
\draw[densely dotted] (4,-2) -- (4,-5);
\draw
  (4,-4) coordinate (a)
  (4,-2) coordinate (b)
  (6,-4) coordinate (c)
  pic[dashed, "$\theta_0$", draw=black, ->, angle radius=1.5cm]{angle=a--b--c};

\draw[dotted] (4,-2) -- (6,-4);
\draw
  (6,-4) coordinate (a)
  (4,-2) coordinate (b)
  (7,-2) coordinate (c)
  pic["$\theta_i$", draw=black, ->, angle radius=1.5cm]{angle=a--b--c};

\draw
  (0,-4) coordinate (a)
  (0,0) coordinate (b)
  (4,-2) coordinate (c)
  pic["$\theta_B$", draw=black, ->, angle radius=1.5cm]{angle=a--b--c};

% masses
\draw[fill=white] (7,-2) circle (0.5cm) node[label={[label distance=0.2cm]east:{$(x_i, y_i)$}}]{$m_i$};

\end{tikzpicture}

\bigbreak
\noindent
This represents one half of a typical booster. It assumes that the mass difference between the booster arm and the gondolas is big enough that the effect of the gondolas on the booster arm is negligible.

\noindent
Furthermore, since we don't know the mass distribution of the gondola we model $n$ masses being connected to the booster arm. Their positions are defined relative to $\theta_0$ so their movement can be tracked using one equation.

\section{Assumptions}
\begin{itemize}
    \item Point masses
    \item Massless, rigid rods
    \item $\dot{\theta_B}$ and $\ddot{\theta_B}$ are not affected by the motion of the masses. In practice, $\dot{\theta_B}$ will be constant and $\ddot{\theta_B}$ will be $0$, but we'll not assume that so they stay present in the final equations
    \item Gravity is present
\end{itemize}

\section{Trigonometric Identities}
\begin{equation} \label{eq:ti_coscos_sinsin}
\cos a \cos b +- \sin a \sin b = \cos{a -+ b}
\end{equation}
\begin{equation} \label{eq:ti_sincos_cossin}
\sin a \cos b +- \cos a \sin b = \sin{a +- b}
\end{equation}
\begin{equation} \label{eq:ti_sin2_cos2}
\sin^2 a + \cos^2 a = 1
\end{equation}

\section{Kinematic Constrains}
\begin{equation} \label{eq:x_i}
x_i = r_B \sin \theta_B + r_i \sin{(\theta_0 + \theta_i)}
\end{equation}
\begin{equation} \label{eq:y_i}
y_i = -r_B \cos \theta_B - r_i \cos{(\theta_0 + \theta_i)}
\end{equation}

\section{Velocities}
\begin{equation} \label{eq:x_i_dot}
\dot{x_i} = r_B \cos \theta_B \dot{\theta_B} + r_i \cos{(\theta_0 + \theta_i)} \dot{\theta_0}
\end{equation}
\begin{equation} \label{eq:y_i_dot}
\dot{y_i} = r_B \sin \theta_B \dot{\theta_B} + r_i \sin{(\theta_0 + \theta_i)} \dot{\theta_0}
\end{equation}

\section{Accelerations}
\begin{equation} \label{eq:x_i_ddot}
\ddot{x_i} =
    - r_B \sin \theta_B \dot{\theta_B}^2
    + r_B \cos \theta_B \ddot{\theta_B}
    - r_i \sin{(\theta_0 + \theta_i)} \dot{\theta_0}^2
    + r_i \cos{(\theta_0 + \theta_i)} \ddot{\theta_0}
\end{equation}
\begin{equation} \label{eq:y_i_ddot}
\ddot{y_i} = 
    r_B \cos \theta_B \dot{\theta_B}^2
    r_B \sin \theta_B \ddot{\theta_B}
    + r_i \cos{(\theta_0 + \theta_i)} \dot{\theta_0}^2
    + r_i \sin{(\theta_0 + \theta_i)} \ddot{\theta_0}
    + g
\end{equation}

\section{Potential Energy}
\begin{equation} \label{eq:V_base}
V = g \sum_{i=1}^{n}(m_i y_i)
\end{equation}
Substitute \ref{eq:y_i} into \ref{eq:V_base}:
\begin{equation} \label{eq:V}
V = g \sum_{i=1}^{n}(-m_i r_B \cos \theta_B - m_i r_i \cos{(\theta_0 + \theta_i)})
\end{equation}

\section{Kinetic Energy}
Base formula:
\begin{equation}
T = \frac{1}{2} \sum_{i=1}^{n}(m_i v_i^2)
\end{equation}
Substitute $v_1$, $v_2$:
\begin{equation} \label{eq:T_base}
T = \frac{1}{2} \sum_{i=1}^{n}(m_i \dot{x_i}^2 + m_i \dot{y_i}^2)
\end{equation}
Substitute \ref{eq:x_i_dot}, \ref{eq:y_i_dot} into \ref{eq:T_base}:
\begin{align}
T &= \frac{1}{2} \sum_{i=1}^{n}(
     m_i (r_B \cos \theta_B \dot{\theta_B} 
   + r_i \cos{(\theta_0 + \theta_i)} \dot{\theta_0})^2
   + m_i (r_B \sin \theta_B \dot{\theta_B}
   + r_i \sin{(\theta_0 + \theta_i)} \dot{\theta_0})^2
)
\end{align}
expand the squares:
\begin{align}
T = \frac{1}{2} \sum_{i=1}^{n}(
    &m_i r_B^2 \cos^2 \theta_B \dot{\theta_B}^2
  + m_i r_i^2 \cos^2{(\theta_0 + \theta_i)} \dot{\theta_0}^2
  + 2 m_i r_B \cos \theta_B \dot{\theta_B} r_i \cos{(\theta_0 + \theta_i)} \dot{\theta_0} \nonumber \\
  + &m_i r_B^2 \sin^2 \theta_B \dot{\theta_B}^2
  + m_i r_i^2 \sin^2{(\theta_0 + \theta_i)} \dot{\theta_0}^2
  + 2 m_i r_B \sin \theta_B \dot{\theta_B} r_i \sin{(\theta_0 + \theta_i)} \dot{\theta_0}
)
\end{align}
Simplify using \ref{eq:ti_coscos_sinsin} and \ref{eq:ti_sin2_cos2}:
\begin{align} \label{eq:T}
T = \frac{1}{2} \sum_{i=1}^{n}(
    &m_i r_B^2 \dot{\theta_B}^2
  + m_i r_i^2 \dot{\theta_0}^2
  + 2 m_i r_B \dot{\theta_B} r_i \dot{\theta_0} \cos(\theta_B - \theta_0 - \theta_i)
)
\end{align}

\section{Lagrangian}
\begin{equation} \label{eq:L_base}
L = T - V
\end{equation}
Substitute \ref{eq:T}, \ref{eq:V} into \ref{eq:L_base}:
\begin{align}
L &= \frac{1}{2} \sum_{i=1}^{n}(
         m_i r_B^2 \dot{\theta_B}^2 + m_i r_i^2 \dot{\theta_0}^2
       + 2 m_i r_B \dot{\theta_B} r_i \dot{\theta_0} \cos(\theta_B - \theta_0 - \theta_i)
     ) \nonumber \\
  &- g \sum_{i=1}^{n}(-m_i r_B \cos \theta_B - m_i r_i \cos{(\theta_0 + \theta_i)})
\end{align}
Rearrange:
\begin{align}
L = \sum_{i=1}^{n}(
        &\frac{1}{2} m_i r_B^2 \dot{\theta_B}^2
      + \frac{1}{2} m_i r_i^2 \dot{\theta_0}^2
      + m_i r_B \dot{\theta_B} r_i \dot{\theta_0} \cos(\theta_B - \theta_0 - \theta_i) \nonumber \\
      + &g m_i r_B \cos \theta_B + g m_i r_i \cos{(\theta_0 + \theta_i)}
    )
\end{align}

\section{Lagrange's Equations}
\begin{equation} \label{eq:LE_base}
\frac{d}{dt} \left(\frac{\partial L}{\partial \dot{q}_j}\right) - \frac{\partial L}{\partial q_j} = 0
\end{equation}

\pagebreak
\subsection{$\theta_0$}
\begin{align}
\frac{\partial L}{\partial \dot{\theta_0}} &= 
    \sum_{i=1}^{n}(
        m_i r_i^2 \dot{\theta_0}
      + m_i r_B \dot{\theta_B} r_i \cos{(\theta_B - \theta_0 - \theta_i)}
    )
\end{align}

\begin{align}
\frac{d}{dt} \left(\frac{\partial L}{\partial \dot{\theta_0}}\right) =
    \sum_{i=1}^{n}(
        &m_i r_i^2 \ddot{\theta_0}
      + m_i r_B \ddot{\theta_B} r_i \cos{(\theta_B - \theta_0 - \theta_i)} \nonumber \\
      - &m_i r_B \dot{\theta_B} r_i \sin{(\theta_B - \theta_0 - \theta_i)} (\dot{\theta_B} - \dot{\theta_0})    ) \label{eq:LE_left_t0}
\end{align}

\begin{align}
\frac{\partial L}{\partial \theta_0} &=
    \sum_{i=1}^{n}(
        m_i r_B \dot{\theta_B} r_i \dot{\theta_0} \sin{(\theta_B - \theta_0 - \theta_i)}
      - g m_i r_i \sin{(\theta_0 + \theta_i)}
    ) \label{eq:LE_right_t0}
\end{align}

\bigskip
\noindent
Substitute \ref{eq:LE_left_t0}, \ref{eq:LE_right_t0} into \ref{eq:LE_base}:
\begin{align}
&\sum_{i=1}^{n}(
    m_i r_i^2 \ddot{\theta_0}
  + m_i r_B \ddot{\theta_B} r_i \cos{(\theta_B - \theta_0 - \theta_i)}
  - m_i r_B \dot{\theta_B} r_i \sin{(\theta_B - \theta_0 - \theta_i)} (\dot{\theta_B} - \dot{\theta_0})
) \nonumber \\
- &\sum_{i=1}^{n}(
    m_i r_B \dot{\theta_B} r_i \dot{\theta_0} \sin{(\theta_B - \theta_0 - \theta_i)}
  - g m_i r_i \sin{(\theta_0 + \theta_i)}
) \nonumber \\
= &0
\end{align}
Rearrange:
\begin{align}
\sum_{i=1}^{n}(
    &m_i r_i^2 \ddot{\theta_0}
  + m_i r_B \ddot{\theta_B} r_i \cos{(\theta_B - \theta_0 - \theta_i)}
  - m_i r_B \dot{\theta_B} r_i \sin{(\theta_B - \theta_0 - \theta_i)} (\dot{\theta_B} - \dot{\theta_0})) \nonumber \\
  - &m_i r_B \dot{\theta_B} r_i \dot{\theta_0} \sin{(\theta_B - \theta_0 - \theta_i)}
  + g m_i r_i \sin{(\theta_0 + \theta_i)}
) \nonumber \\
= 0
\end{align}
Simplify:
\begin{align}
\sum_{i=1}^{n}(
    &m_i r_i^2 \ddot{\theta_0} + m_i r_B \ddot{\theta_B} r_i \cos{(\theta_B - \theta_0 - \theta_i)} \nonumber \\
  - &m_i r_B \dot{\theta_B}^2 r_i \sin{(\theta_B - \theta_0 - \theta_i)}
  + g m_i r_i \sin{(\theta_0 + \theta_i)}
) \nonumber \\
= &0
\end{align}
Separate sums for separate forces:
\begin{align}
  &\ddot{\theta_0} \sum_{i=1}^{n}(m_i r_i^2)
+ r_B \ddot{\theta_B} \sum_{i=1}^{n}(m_i r_i \cos{(\theta_B - \theta_0 - \theta_i)}) \nonumber \\
- &r_B \dot{\theta_B}^2 \sum_{i=1}^{n}(m_i r_i \sin{(\theta_B - \theta_0 - \theta_i)})
+ g \sum_{i=1}^{n}(m_i r_i \sin{(\theta_0 + \theta_i)}))
= 0
\end{align}

\pagebreak
\section{State space}
solve for $\ddot{\theta_0}$:
\begin{equation}
\ddot{\theta_0} =
    \frac{
        \splitfrac{
            \splitfrac{
                - r_B \ddot{\theta_B} \sum_{i=1}^{n}(m_i r_i \cos{(\theta_B - \theta_0 - \theta_i)})
            } {
                + r_B \dot{\theta_B}^2 \sum_{i=1}^{n}(m_i r_i \sin{(\theta_B - \theta_0 - \theta_i)})
            }
        } {
            - g \sum_{i=1}^{n}(m_i r_i \sin{(\theta_0 + \theta_i)})
        }
    } {
        \sum_{i=1}^{n}(m_i r_i^2)
    } \label{eq:thetadd}
\end{equation}
rewrite as state space:
\begin{equation}
\dot{\overrightarrow{y}} = \frac{d}{dt}
\left\{\!
\begin{array}{c}
  \theta_0 \\
  \dot{\theta_0}
\end{array}
\!\right\} = 
\left\{\!
\begin{array}{c}
  \dot{\theta_0} \\
  \ddot{\theta_0} \\
\end{array}
\!\right\}
\end{equation}

\section{Sensor data}
Like all the masses, the Sensor is located at $(\theta_s, r_s)$ which is defined relative to $\theta_0$ in the gondola frame.

\subsection{Accelerometer}
\subsubsection{$x$}
Rotate $\ddot{x_s}$ into inertial frame:
\begin{align}
\ddot{x_s}_{inertial} = \ddot{x_s} \cos \theta_s - \ddot{y_s} \sin \theta_s 
\end{align}
Substitute:
\begin{align}
\ddot{x_s}_{inertial} &= \nonumber \\
    &- r_B \sin \theta_B \cos{(-\theta_0 - \theta_s)} \dot{\theta_B}^2
     + r_B \cos \theta_B \cos{(-\theta_0 - \theta_s)} \ddot{\theta_B} \nonumber \\
    &- r_s \sin{(\theta_0 + \theta_s)} \cos{(-\theta_0 - \theta_s)} \dot{\theta_0}^2
     + r_s \cos{(\theta_0 + \theta_s)} \cos{(-\theta_0 - \theta_s)} \ddot{\theta_0} \nonumber \\
    &- r_B \cos \theta_B \sin{(-\theta_0 - \theta_s)} \dot{\theta_B}^2
     - r_B \sin \theta_B \sin{(-\theta_0 - \theta_s)} \ddot{\theta_B} \nonumber \\
    &- r_s \cos{(\theta_0 + \theta_s)} \sin{(-\theta_0 - \theta_s)} \dot{\theta_0}^2
     - r_s \sin{(\theta_0 + \theta_s)} \sin{(-\theta_0 - \theta_s)} \ddot{\theta_0} \nonumber \\
    &- g \sin{(-\theta_0 - \theta_s)}
\end{align}
Simplify using \ref{eq:ti_coscos_sinsin} and \ref{eq:ti_sincos_cossin}:
\begin{align}
\ddot{x_s}_{inertial} &= \nonumber \\
    &- r_B \dot{\theta_B}^2 \sin{(\theta_B - \theta_0 - \theta_s)}
     + r_B \ddot{\theta_B} \cos{(\theta_B -\theta_0 - \theta_s)} \nonumber \\
    &- r_s \dot{\theta_0}^2 \sin{(\theta_0 + \theta_s -\theta_0 - \theta_s)}
     + r_s \ddot{\theta_0} \cos{(\theta_0 + \theta_s -\theta_0 - \theta_s)} \nonumber \\
    &- g \sin{(-\theta_0 - \theta_s)}
\end{align}
Remove 0/1-terms:
\begin{align}
\ddot{x_s}_{inertial} &= \nonumber \\
    &- r_B \dot{\theta_B}^2 \sin{(\theta_B - \theta_0 - \theta_s)}
     + r_B \ddot{\theta_B} \cos{(\theta_B -\theta_0 - \theta_s)} \nonumber \\
    &+ r_s \ddot{\theta_0} \nonumber \\
    &- g \sin{(-\theta_0 - \theta_s)}
\end{align}

\subsubsection{$y$}
Rotate $\ddot{y_s}$ into inertial frame:
\begin{align}
\ddot{y_s}_{inertial} = \ddot{x_s} \sin \theta_s + \ddot{y_s} \cos \theta_s 
\end{align}
\begin{align}
\ddot{y_s}_{inertial} &= \nonumber \\
    &- r_B \sin \theta_B \sin{(-\theta_0 - \theta_s)} \dot{\theta_B}^2
     + r_B \cos \theta_B \sin{(-\theta_0 - \theta_s)} \ddot{\theta_B} \nonumber \\
    &- r_s \sin{(\theta_0 + \theta_s)} \sin{(-\theta_0 - \theta_s)} \dot{\theta_0}^2
     + r_s \cos{(\theta_0 + \theta_s)} \sin{(-\theta_0 - \theta_s)} \ddot{\theta_0} \nonumber \\
    &+ r_B \cos \theta_B \cos{(-\theta_0 - \theta_s)} \dot{\theta_B}^2
     + r_B \sin \theta_B \cos{(-\theta_0 - \theta_s)} \ddot{\theta_B} \nonumber \\
    &+ r_s \cos{(\theta_0 + \theta_s)} \cos{(-\theta_0 - \theta_s)} \dot{\theta_0}^2
     + r_s \sin{(\theta_0 + \theta_s)} \cos{(-\theta_0 - \theta_s)} \ddot{\theta_0} \nonumber \\
    &+ g \cos{(-\theta_0 - \theta_s)}
\end{align}
Simplify using \ref{eq:ti_coscos_sinsin} and \ref{eq:ti_sincos_cossin}:
\begin{align}
\ddot{y_s}_{inertial} &= \nonumber \\
    &- r_B \dot{\theta_B}^2 \cos{(\theta_B - \theta_0 - \theta_s)}
     + r_B \ddot{\theta_B} \sin{(\theta_B - \theta_0 - \theta_s)} \nonumber \\
    &+ r_s \dot{\theta_0}^2 \cos{(\theta_0 + \theta_s - \theta_0 - \theta_s)}
     + r_s \ddot{\theta_0} \sin{(\theta_0 + \theta_s - \theta_0 - \theta_s)} \nonumber \\
    &+ g \cos{(-\theta_0 - \theta_s)}
\end{align}
Remove 0/1-terms:
\begin{align}
\ddot{y_s}_{inertial} &= \nonumber \\
    &- r_B \dot{\theta_B}^2 \cos{(\theta_B - \theta_0 - \theta_s)}
     + r_B \ddot{\theta_B} \sin{(\theta_B - \theta_0 - \theta_s)} \nonumber \\
    &+ r_s \dot{\theta_0}^2 \nonumber \\
    &+ g \cos{(-\theta_0 - \theta_s)}
\end{align}

\subsubsection{Vector in ENU coordinates}
\begin{equation}
Accelerometer_{ENU} = \left[ \begin{array}{c}
0 \\
\ddot{x_s}_{inertial} \\
\ddot{y_s}_{inertial}
\end{array} \right]
\end{equation}

\subsection{Gyroscope}
Since $\theta_0$ is aligned with the sensor:
\begin{equation}
Gyroscope_{ENU} = \left[ \begin{array}{c}
\dot{\theta_0} \\
0 \\
0
\end{array} \right]
\end{equation}

\section{Simulate multi-mass booster using single-mass equations}
Doing so allows us to determine $\theta_c$ and $r_c$ for the center of mass using careful analysis of IMU recordings without ever knowing the exact mass distribution of the gondola.
The math in this section serves as a proof that this is in fact possible and to find formulas for $\theta_c$ and $r_c$ to then compare the two simulations to see if they really are identical.

\bigskip
\noindent
\textbf{The interesting finding here is that while - as expected - $\theta_c$ points toward the center of mass as calculated using cartesian coordinates, $r_c$ is actually different from what the center-of-mass calculation tells us.}

\subsection{Convert from cartesian to polar coordinates}
\begin{align}
r &= \sqrt{x^2 + y^2} \\
\theta &= \arctan{\left(\frac{x}{-y}\right)}
\end{align}

\subsection{Convert from polar to cartesian coordinates}
\begin{align}
x &= r \sin \theta \\
y &= -r \cos \theta
\end{align}

\subsection{$\theta_c$}
total mass:
\begin{equation}
m_c = \sum_{i=1}^{n}m_i
\end{equation}
cartesian center of mass:
\begin{align}
R_x &= \frac{1}{m_c} \sum_{i=1}^{n}(m_i r_i \sin \theta_i) \\
R_y &= \frac{1}{m_c} \sum_{i=1}^{n}(- m_i r_i \cos \theta_i) 
\end{align}
convert to polar angle(\textbf{doing this for the radius would yield a wrong result}):
\begin{align}
\theta_c &= \arctan{\left(\frac{R_x}{-R_y}\right)} \\
\theta_c &= \arctan{\left(\frac{
    \sum_{i=1}^{n}(m_i r_i \sin \theta_i)
} {
    \sum_{i=1}^{n}(m_i r_i \cos \theta_i)
}\right)}
\end{align}
Simplify \ref{eq:thetadd} for the special case of a single mass:
\begin{align}
\ddot{\theta_0}_c =
    \frac{
        - r_B \ddot{\theta_B} \cos{(\theta_B - \theta_0 - \theta_c)}
        + r_B \dot{\theta_B}^2 \sin{(\theta_B - \theta_0 - \theta_c)}
        - g \sin{(\theta_0 + \theta_c)}
    } {
        r_c
    } \label{eq:theta0cdd}
\end{align}
If we put the second and the third force in a reference frame relative to $\theta_0$, we can see that they both follow the same scheme with $F$ being some force:
\begin{equation}
\frac{F}{r_c} \sin \theta_c
\end{equation}
Do the same for \ref{eq:thetadd}:
\begin{equation}
\frac{F}{\sum_{i=1}^{n}(m_i r_i^2)} \sum_{i=1}^{n}(m_i r_i \sin \theta_i)
\end{equation}
We want them to be equal, so  let's do that:
\begin{equation}
\frac{F}{r_c} \sin \theta_c = 
    \frac{F}{\sum_{i=1}^{n}(m_i r_i^2)} \sum_{i=1}^{n}(m_i r_i \sin \theta_i)
\end{equation}
Solve for $\theta_c$:
\begin{equation}
\sin \theta_c = 
    \frac{
        r_c \sum_{i=1}^{n}(m_i r_i \sin \theta_i)
    } {
        \sum_{i=1}^{n}(m_i r_i^2)
    }
\end{equation}
\begin{equation}
\theta_c = 
\arcsin{\left(
    \frac{
        r_c \sum_{i=1}^{n}(m_i r_i \sin \theta_i)
    } {
        \sum_{i=1}^{n}(m_i r_i^2)
    }
\right)}
\end{equation}
We now have two formulas for $\theta_c$, one defined by the center of mass, and one by setting the forces for single- and multi-mass boosters equal.
Now we can set those two formulas equal as well:
\begin{equation}
\arcsin{\left(
    \frac{
        r_c \sum_{i=1}^{n}(m_i r_i \sin \theta_i)
    } {
        \sum_{i=1}^{n}(m_i r_i^2)
    }
\right)}
= \arctan{\left(\frac{
    \sum_{i=1}^{n}(m_i r_i \sin \theta_i)
} {
    \sum_{i=1}^{n}(m_i r_i \cos \theta_i)
}\right)}
\end{equation}
Now solve for $r_c$. Start by applying $\sin$:
\begin{equation}
\frac{
    r_c \sum_{i=1}^{n}(m_i r_i \sin \theta_i)
} {
    \sum_{i=1}^{n}(m_i r_i^2)
}
= \sin{\left( \arctan{\left( \frac{
    \sum_{i=1}^{n}(m_i r_i \sin \theta_i)
} {
    \sum_{i=1}^{n}(m_i r_i \cos \theta_i)
}\right)} \right)}
\end{equation}
apply $\sin {\left( \arctan x \right)} = \frac{x}{\sqrt{x^2 + 1}}$:
\begin{equation}
\frac{
    r_c \sum_{i=1}^{n}(m_i r_i \sin \theta_i)
} {
    \sum_{i=1}^{n}(m_i r_i^2)
}
= \frac{
    \sum_{i=1}^{n}(m_i r_i \sin \theta_i)
} {
    \sum_{i=1}^{n}(m_i r_i \cos \theta_i)
    \sqrt{\frac{
        \left(\sum_{i=1}^{n}(m_i r_i \sin \theta_i)\right)^2
    } {
        \left(\sum_{i=1}^{n}(m_i r_i \cos \theta_i)\right)^2
    } + 1 }
}
\end{equation}
Pull in $\sum_{i=1}^{n}(m_i r_i \cos \theta_i)$ by squaring it:
\begin{equation}
\frac{
    r_c \sum_{i=1}^{n}(m_i r_i \sin \theta_i)
} {
    \sum_{i=1}^{n}(m_i r_i^2)
}
= \frac{
    \sum_{i=1}^{n}(m_i r_i \sin \theta_i)
} {
    \sqrt{\frac{
        \left(\sum_{i=1}^{n}(m_i r_i \sin \theta_i)\right)^2 \left(\sum_{i=1}^{n}(m_i r_i \cos \theta_i)\right)^2
    } {
        \left(\sum_{i=1}^{n}(m_i r_i \cos \theta_i)\right)^2
    } + \left(\sum_{i=1}^{n}(m_i r_i \cos \theta_i)\right)^2 }
}
\end{equation}
Simplify:
\begin{equation}
\frac{
    r_c \sum_{i=1}^{n}(m_i r_i \sin \theta_i)
} {
    \sum_{i=1}^{n}(m_i r_i^2)
}
= \frac{
    \sum_{i=1}^{n}(m_i r_i \sin \theta_i)
} {
    \sqrt{
        \left(\sum_{i=1}^{n}(m_i r_i \sin \theta_i)\right)^2
      + \left(\sum_{i=1}^{n}(m_i r_i \cos \theta_i)\right)^2
    }
}
\end{equation}
move the fraction from the left to the right:
\begin{equation}
r_c = \frac{
    \sum_{i=1}^{n}(m_i r_i^2)
} {
    \sqrt{
        \left(\sum_{i=1}^{n}(m_i r_i \sin \theta_i)\right)^2
      + \left(\sum_{i=1}^{n}(m_i r_i \cos \theta_i)\right)^2
    }
}
\end{equation}
If we do the same calculation with the first force in \ref{eq:theta0cdd} which multiplies by $\cos$ instead of $\sin$ we get the exact same result. This means that $r_c$ and $\theta_c$ can be used to simulate any multi-mass booster.

\end{document}
